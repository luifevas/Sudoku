\documentclass[12pt,oneside]{book}
\usepackage{geometry} % See geometry.pdf to learn the layout options. There are lots.
\geometry{a4paper} % ... or a4paper or a5paper or ...
%\geometry{landscape} % Activate for for rotated page geometry
%\usepackage[parfill]{parskip} % Activate to begin paragraphs with an empty line rather than an indent
\usepackage{graphicx}	% Use pdf, png, jpg, or epsß with pdflatex; use eps in DVI mode
% TeX will automatically convert eps --> pdf in pdflatex
\usepackage{amssymb}

\usepackage[spanish]{babel}	% Permite que partes automáticas del documento aparezcan en castellano.
\usepackage[utf8]{inputenc}	% Permite escribir tildes y otros caracteres directamente en el .tex
\usepackage[T1]{fontenc}	% Asegura que el documento resultante use caracteres de una fuente apropiada.

\usepackage{hyperref}	
\title{Sudoku}
\author{Luis Vásquez y Luis Caviedes}

\begin{document}
\maketitle
\tableofcontents
\chapter{Introducción}
Proyecto elaborado en C++ en el framework Qt. Es un juego de Sudoku relativamente sencillo, pero que nos forzó a poner nuestras habilidades adquiridas a través de la carrera para poder presentar este proyecto.
\chapter{Instrucciones}
\section{Instrucciones}

\begin{figure}[htbp]
\begin{center}
\includegraphics[width=.60\textwidth]{./imagenes/1.jpg}
\caption{Ventana Principal}
\label{Ventana Principal}
\end{center}
\end{figure}
El usuario comienza en la ventana principal, donde puede escoger cualquiera de las opciones presentadas. Para empezar el juego, le deberá dar click en jugar.


\begin{figure}[htbp]
\begin{center}
\includegraphics[width=.60\textwidth]{./imagenes/2.jpg}
\caption{Ventana de Elección}
\label{Ventana de Elección}
\end{center}
\end{figure}
En esta ventana ingresará la dificultad y el nombre. La dificultad esta dada por un número (dado por la dificultad), que al ingresar al algoritmo es el la cantidad de números que se va a extraer por recuadro.

\begin{figure}[htbp]
\begin{center}
\includegraphics[width=.60\textwidth]{./imagenes/3.jpg}
\caption{Ventana del Juego}
\label{Ventana del Juego}
\end{center}
\end{figure}

Ya en esta ventana el usuario puede jugar. Se recuerda que no se podrá acceder a la funcionalidad del ranking si se carga una partida , se usa la funcionalidad pista o se usa las funcionalidades Jugadas Inválidas e Incorrectas.

Al presionar el boton pista una casilla vacía tomará el valor correcto segun la solución del sudoku.

\begin{figure}[htbp]
\begin{center}
\includegraphics[width=.60\textwidth]{./imagenes/3.jpg}
\caption{Jugadas Inválidas}
\label{Jugadas Inválidas}
\end{center}
\end{figure}

Al acceder a la funcionalidad de las jugadas inválidas, se verifica el estado actual del sudoku, y jugadas incorrectas se verifica los números incorrectos con respecto a la solución del sudoku.

El reloj en la parte inferior indica el tiempo, el cuál es el puntaje del jugador.
\chapter{Funcionalidades}
\section{Ranking}
Esta funcionalidad es accecida mediante la primera pantalla del programa, dandole click al botón ranking.
\begin{figure}[htbp]
\begin{center}
\includegraphics[width=.60\textwidth]{./imagenes/5.jpg}
\caption{Pantalla Ranking}
\label{Pantalla Ranking}
\end{center}
\end{figure}


Esta funcionalidad se la llevo a cabo usando un widget en Qt especial, que nos permitió armar una tabla con los nombres de los jugadores. El ranking es guardado en un archivo de texto ranking.txt, y es cargado en tiempo real a la tabla. Al finalizar el juego de forma satisfactoria el nombre del jugador y el puntaje (número de segundos) son guardados en este archivo.



\end{document}